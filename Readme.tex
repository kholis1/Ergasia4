\documentclass[10pt,a4]{article}
\usepackage[english,greek]{babel}
\usepackage{epsfig}
\usepackage{epsfig}
\usepackage{ucs}
\usepackage[utf8x]{inputenc}
\usepackage{url}

\textwidth      6.5in
\textheight     8.7in
\oddsidemargin  0.0cm
\evensidemargin 0.0cm
\parindent      0pt              
\topmargin      0.1in
\headsep        20pt

\parskip        10pt
\parindent      0pt

\newcommand{\gr}{\greektext}
\newcommand{\la}{\latintext}


\newenvironment{pkl}{%
\begin{itemize}%
\vspace{-1.8\topsep}%
\setlength\itemsep{-\parskip}%
\setlength\parsep{0in}%
}{%
\vspace{-1.1\topsep}%
\end{itemize}}

 
\newenvironment{enu}{%
\begin{enumerate}%
\vspace{-1.8\topsep}%
\setlength\itemsep{-\parskip}%
\setlength\parsep{0.10in}%
}{%
\vspace{-\topsep}%
\end{enumerate}}      

\begin{document}

\thispagestyle{empty}

\vspace*{-2.0cm}

{\large \sf
\begin{center}
{\large Λειτουργικά Συστήματα}\\
Eργασία 4 \\
Νίκος Ευτυχίου \\
\la{sdi1800221}\\
\end{center}
}

\baselineskip=1.1\baselineskip 

\begin{itemize}
\item
	{\sf Γενικες πληροφορίες σχετικά με τον πρόγραμμα:}\\ \\
	Το πρόγραμμα επιλέχθηκε να δομηθεί σε 7  αρχεία για την ευκολότερη κατανόηση και σαφήνεια του
	κώδικα.\\ \\
	Για να γίνει {\la compile} το πρόγραμμα χρησιμοποιήστε την εντολή {\la make}.\\ \\
	Το πρόγραμμα {\la quic} πρέπει  να δεχθεί σαν είσοδο από την γραμμή εντολών την παράμετρο {\la "-f source dest"}
	όπου {\la source} και {\la dest} ο φάκελος για αντιγραφή και το μονοπάτι που θα γίνει η αντιγραφή αντίστοιχα.
	Επίσης στο πρόγραμμα μπορούν να δοθούν οι παραμέτροι {\la "-d" , "-l" , "-v"} όπου διαγράφει περιττά αρχεία 
	που υπάρχουν στο {\la dest} αλλά όχι στο {\la source}, λαμβάνει υπόψην τα {\la links} που ενδέχεται να υπάρχουν
	και εκτυπώνει κάποιες επιπλέον πληροφορίες σχετικά με το τι συμβαίνει στο πρόγραμμα.\\
	
	
	
	
	{\sf Αρχείο {\la quic.c:}}\\ \\
	H {\la main} συνάρτηση αρχικά καλεί την συνάρτηση {\la parametersCheck} για να ελέγξει αν έχουν δοθεί σωστά
	τα ορίσματα στο πρόγραμμα. Στη συνέχεια μέσο της {\la getParametersValues} παίρνει τα ορίσματα που 
	έχουν δοθεί στο πρόγραμμα, με την βοήθεια της {\la opendir} ελέγχει αν το μονοπάτι που έχει δοθεί για αντιγραφή
	υπάρχει αλλιώς το δημιουργεί με την {\la CheckDest}. Έπειτα αν υπάρχει η {\la "-d" } τότε καλείται η 
	{\la CheckDeletedFiles} για να σβήσει όποια αρχεία ή και φακέλους δεν χρειάζονται. Μετά με την {\la CopyFile} 
	ξεκινά η κύρια λειτουργεία του προγράμματος, δηλαδή η αντιγραφή των αρχείων.Τέλος εκτυπώνονται πληροφορίες
	σχετικές με την {\la quic}.
	
	{\sf Αρχείο {\la delete.c:}}\\ \\
	Συνάρτηση {\la void CheckDeletedFiles(char *origin, char *dest, int verbose)}\\
	Η συνάρτηση αρχικά ανοίγει το {\la directory destination} και σε ένα βρόγχο διαβάζει όλα τα αρχεία που βρίσκονται
	στο {\la destination}. Σε ένα δεύτερο βρόγχο ανοίγουμε το {\la directory source} για να διαβάσουμε επίσης όλα τα
	αρχεία του και να τα συγκρίνουμε με τα αρχεία στο {\la destination}. Αν δύο αρχεία έχουν ίδιο όνομα τότε γίνεται έλεγχος
	αν είναι {\la directories} και τα δύο, όπου δημιουργείται το νέο μονοπάτι για {\la source} και {\la destination} για να 
	κλειθεί αναδρομικά η συνάρτηση και τα ελέγξει το {\la subdirectory}. Αν τα δύο αρχεία έχουν διαφορετικό τύπο ή αν ένα
	αρχείο υπάρχει μόνο στο {\la destination directory} τότε διαγράφεται. Αν ο τύπος είναι {\la directory} για διαγραφή τότε
	η {\la RemoveAll} φρονίζει να έχουν διαγραφεί όλα όσα βρίσκοντα μέσα στο {\la directory}.\\
	
	Συνάρτηση  {\la void RemoveAll(char *path, int verbose)}\\
	Η συνάρτηση με τις συναρτήσεις {\la opendir} και {\la readir}  παίρνει όλα τα αρχεία και τα διαγράφει από το {\la directory}.
	Άν υπάρχει κάποιο {\la subdirectory} τότε διαγράφεται αναδρομικά με την {\la RemoveAll} ξανά.
	
	{\sfΑρχείο {\la copy.c}:}\\
	Συνάρτηση {\la int CopyFiles(char *origin, char *dest,char *initialOrigin, char *initialDest, int l, long int *bytes, int verbose)}\\
	Η συνάρτηση αρχικά ανοίγει το {\la directory source} και σε ένα βρόγχο διαβάζει όλα τα αρχεία που βρίσκονται
	στο {\la source}. Σε ένα δεύτερο βρόγχο ανοίγουμε το {\la directory destination} για να διαβάσουμε επίσης όλα τα
	αρχεία του και να τα συγκρίνουμε με τα αρχεία στο {\la source}.\
	
	Αν δύο αρχεία έχουν ίδιο όνομα τότε γίνεται έλεγχος αν είναι {\la directories} και τα δύο, όπου δημιουργείται 
	το νέο μονοπάτι για {\la source} και {\la destination} για να κλειθεί αναδρομικά η συνάρτηση και τα ελέγξει το 
	{\la subdirectory}.\
	
	 Αν δύο ομόνημα αρχεία  έχουν ίδιο τύπο τότε δημιουργούμε το μονοπάτι για τα δύο αρχεία και ελέγχουμε άν έχουν ίδιο
	 μέγεθος και ίδια ημερομηνία τελευταίας τροποίησης. Αν κάποιο από τα δύο δεν συνάδει τότε διαγράφεται το αρχείο
	 από το {\la destination} και καλούμε την {\la Copy} για να δημιουργήσει το νέο αρχείο.\
	 
	 Αν δύο αρχεία δεν έχουν ίδιο τύπο και το αρχείο στο {\la destination} είναι {\la directory} τότε δημιουργούμε
	 το μονοπάτι για τα δύο αρχεία, με την {\la RemoveAll} διαγράφουμε το {\la directory destination} και με την 
	 {\la Copy} αντιγράφουμε το αρχείο.\
	 
	 Αν δύο αρχεία δεν έχουν ίδιο τύπο και το αρχείο στο {\la source} είναι {\la directory} τότε δημιουργούμε
	 το μονοπάτι για τα δύο αρχεία, διαγράφουμε το {\la destination}, δημιουργούμε το {\la directory destination}.
	 και καλούμε αναδρομικά την {\la CopyFiles}.\
	 
	 Aν δύο αρχεία δεν έχουν ίδιο τύπο και δεν είναι {\la directories} τότε δημιουργούμε το μονοπάτι,
	  διαγράφουμε το {\la destination} και με την {\la Copy} αντιγράφουμε το αρχείο.\
	  
	  Τέλος αν το αρχείο δεν υπάρχει στο {\la Destination} τότε δημιουργούμε το νέο μονοπάτι, ελέγχουμε αν είναι
	  {\la directory} και αν είναι δημιουργούμε το νέο {\la directory} και καλούμε την {\la CopyFiles} για να αντιγράψει
	  τα αρχεία. Σε αντίθετη περίπτωση με την {\la Copy} αντιγράφουμε το αρχείο στο {\la destination}  και επιστρέφει
	  τον αριθμό των αρχείων που έχουν ελεγθεί .\\ \\ 
	  
	  Συνάρτηση {\la Copy(char *source, char *dest,char *initialSource,char *initialDest, int l, int verbose))}\\
	  Αρχικά η συνάρτηση ελέγχει αν το {\la source} όπου έχει δωθεί είναι {\la symbolic link} και αν έχει δοθεί
	  η {\la "-l"}.\\
	  Αν ναι, τότε συγκρίνει το {\la absolute path} του {\la source directory} και το {\la absolute path}
	  στο οποίο "δείχνει" το {\la symbolic link}.  Άν το δεύτερο είναι {\la substring} του πρώτου τότε το {\la link}
	  βρίσκεται μέσα στο {\la source} και δημιουργούμε το κατάλληλο μονοπάτι για το {\la destination}.Αν δεν
	  βρίσκεται μέσα στο {\la source} τότε το {\la link} είναι ίδιο. Στη συνέχεια  δημιουργεί το {\la symbolic link}
	  και με την {\la lutimes} εξισώνει το {\la last accessed date} και {\la last modified date} των δύο αυτών
	  {\la links}.\\
	  Aν στο αρχείο υπάρχουν περισσότερα από ένα {\la hardlinks} με τη βοήθεια της {\la CheckHardlink} ελέγχουμε
	  αν όλα τα {\la hardlinks} σε αυτό το {\la inode} βρίσκονται μέσα στο {\la source}. Αν όχι τότε κάνουμε απλά
	  {\la link} το αρχείο. Σε αντίθετη περίπτωση ψάχνουμε αν υπάρχει το {\la link} στο {\la destination}. Αν
	  υπάρχει τότε κάνουμε {\la link} το νέο αρχείο αλλιώς κάνουμε {\la deep copy} στο {\la destination}.\\
	  Σε οποιαδήποτε άλλη περίπτωση κάνουμε {\la deep copy} το αρχείο με την {\la cp}.\\
	  Τέλος με την {\la utimes} εξισώνει το {\la last accessed date} και {\la last modified date} των δύο αυτών αρχείων.
	  
 	 Συνάρτηση  {\la int cp(char *source, char *dest))}\\
 	 H συνάρτηση ανοίγει το {\la source} αρχείο για να το διαβάσει και το {\la destination} αρχείο για να γράψει σε αυτό
 	 ό,τι έχει διαβάσει από το {\la source}.(H συνάρτηση έχει ίδια βάση με την συνάρτηση του κ. Α. Δελή στις διαφάνειες
 	 και έχει τροποποιηθεί κατάλληλα για τους σκοπούς αυτής της εργασίας).\\ \\
 	 
 	 {\sf Αρχείο {\la misc.c:}}\\ \\
 	 
 	  Συνάρτηση {\la int parametersCheck(int argc, char * argv[])}\\
 	  Η συνάρτηση ελέγχει αν τα ορίσματα που έχουν δοθεί στο πρόγραμμα είναι ορθά. \\
 	  
 	   Συνάρτηση {\la void getParametersValues(int *v, int *d, int *l, char **origin, char **dest, int argc, char *argv[])}\\
 	   Η συνάρτηση ελέγχει και παίρνει τα ορίσματα που έχουν δοθεί και τα τοποθετεί σε κατάλληλες μεταβλητές.\\
 	 
	   Συνάρτηση {\la void CheckDest(char *Destination)}\\
	   Η συνάρτηση παίρνει σαν όρισμα ένα {\la string}. Με την {\la opendir} ελέγχει αν αυτό το είναι {\la valid path}.
	   Αν δεν είναι τότε με την {\la strtok} συνάρτηση παίρνει ένα προς ένα τα κομμάτια του {\la Destination} και τροποιόντας
	   τα κατάλληλα ελέγχει αν αυτά είναι {\la valid paths} αν δεν είναι τότε δημιουργεί το {\la path}.\\
	   
 	 Συνάρτηση {\la int CheckHardlink(char *root, int inode, char **pathToFile)}\\
 	 H συνάρτηση με την {\la opendir} και {\la readdir} ελέγχει αν κάποιο αρχείο στο φάκελο {\la root} έχει ίδιο {\la inode}
 	 με το όρισμα {\la inode} Aν ναι τότε ο συνολικός αριθμός των αρχείων με ίδιο {\la inode} αυξάνεται και {\la by reference}
 	 εισάγουμε αυτό το μονοπατι στο {\la pathTofile}.Αν το αρχείο είναι {\la directory} τότε καλούμε την συνάρτηση αναδρομικά.
 	 Τέλος η συνάρτηση επιστρέφει τον συνολικό αριθμό των αρχείων με ίδιο {\la inode}.\\
 	 
 	  Συνάρτηση {\la char *ReturnInodePath(char *initialSource, char *initialDest, int inode)}\\
 	 H συνάρτηση με την {\la opendir} και {\la readdir} ελέγχει αν κάποιο αρχείο στο φάκελο {\la initialSource} έχει ίδιο {\la inode}
 	 με το το όρισμα {\la inode}. Για κάθε αρχείο που ελέγχουμε δημιουργείται το κατάλληλο μονοπατι. Σε περίπτωση όπου
 	 υπάρχει κάποιο {\la subdirectory} η συνάρτηση καλείται αναδρομικά. Η συνάρτηση επιστρέφει το μονοπάτι στο αρχείο
 	 του {\la destination} όπου είναι ίδιο με το αρχείο στο {\la source} και σε περίπτωση όπου δεν βρει τότε επιστρέφει
 	 {\la NULL}.	
	
\end{itemize}


\end{document}
